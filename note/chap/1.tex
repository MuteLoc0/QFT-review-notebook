QFT中用西海岸度规$(1,\bm{-1})$

$$\delta(f(x)) = \sum_{i} \frac{\delta(x - x_i)}{|f'(x_i)|}$$
\nsection{经典场论}
\begin{align*}
    H = \int \dif^3 x \, (\pi \dot{\phi} - \mathcal{L}) \qquad
    \bm{P} = - \int \dif^3 x \, \pi(\bm{x}) \nabla \phi(\bm{x})
\end{align*}
其中共轭动量定义为 $\pi = \frac{\partial \mathcal{L}}{\partial \dot{\phi}}= \dot{\phi}$\ (在$\mathcal{L} = \frac{1}{2} \partial_\mu \phi \partial^\mu \phi - \frac{1}{2} m^2 \phi^2$中)。

\subnsection{运动方程 (EOM)}
\begin{equation*}
    \partial_\mu \left( \frac{\partial \mathcal{L}}{\partial (\partial_\mu \phi)} \right) - \frac{\partial \mathcal{L}}{\partial \phi} = 0 \implies (\partial_\mu \partial^\mu + m^2)\phi = 0 \quad (\text{KG方程})
\end{equation*}
在海森堡绘景下的演化方程为:
\begin{equation*}
    \frac{\dif A}{\dif t} = \ramuno [H, A]
\end{equation*}
$a_{p}|_H= \eu^{iHt} a_p  \eu^{-iHt}= a_p  \eu^{-iE_p t}$

\subnsection{量子化}
化到动量表象中,$ \phi(x) = \int \frac{\dif^3 p}{(2\pi)^3}\eu^{\ramuno p \cdot x}\phi(p) $
K-G 方程化为 $[\frac{\partial^2}{\partial t^2} + (|\bm{p}|^2 + m^2)] \phi_{\bm{p}}(t) = 0$

\begin{align*}
    \phi(x) &= \int \frac{\dif^3 p}{(2\pi)^3 \sqrt{2E_p}} \left( a_{\bm{p}} \eu^{-\ramuno p \cdot x} + a_{\bm{p}}^\dagger \eu^{\ramuno p \cdot x} \right) \\
    \pi(x) &= \int \frac{\dif^3 p}{(2\pi)^3} (-\ramuno)\sqrt{\frac{E_p}{2}} \left( a_{\bm{p}} \eu^{-\ramuno p \cdot x} - a_{\bm{p}}^\dagger \eu^{\ramuno p \cdot x} \right)
\end{align*}
其中 $p \cdot x = E_p t - \bm{p}\cdot\bm{x}$。\,
等时正则对易关系:
\begin{align*}
    [a_{\bm{p}}, a_{\bm{q}}^\dagger] &= (2\pi)^3 \delta^{(3)}(\bm{p}-\bm{q}) \\
    [\phi(\bm{x}), \pi(\bm{y})] &= \ramuno \delta^{(3)}(\bm{x}-\bm{y})
\end{align*}
\begin{align*}
    H &= \int \dif^3 x \left[ \frac{1}{2}\pi^2 + \frac{1}{2}(\nabla \phi)^2 + \frac{1}{2}m^2\phi^2 \right] 
       = \int \frac{\dif^3 p}{(2\pi)^3} E_p \left( a_{\bm{p}}^\dagger a_{\bm{p}} + \frac{1}{2}[a_{\bm{p}}, a_{\bm{p}}^\dagger] \right) \\
    \bm{P} &= \int \frac{\dif^3 p}{(2\pi)^3} \bm{p} \, a_{\bm{p}}^\dagger a_{\bm{p}}
\end{align*}

\begin{center}
    \small{注:张量指标记法 $x_\mu = g_{\mu\nu} x^\nu$,矩阵转置满足 $(\Lambda^T)_\nu^\mu = \Lambda^\mu_\nu$(前行后列)。}
\end{center}

\subnsection{诺特定理}
考虑场变换 $\phi(x) \to \phi(x) + \alpha \Delta \phi(x)$,导致拉格朗日密度变化为全散度项:
\begin{equation*}
    \mathcal{L}(x) \to \mathcal{L}(x) + \alpha \partial_\mu \mathcal{J}^\mu(x)
\end{equation*}
定义诺特流(Noether Current)$j^\mu$:
\begin{equation*}
    j^\mu(x) = \frac{\partial \mathcal{L}}{\partial(\partial_\mu \phi)} \Delta \phi - \mathcal{J}^\mu(x)
\end{equation*}
由运动方程可证该流是守恒的:
\begin{equation*}
    \partial_\mu j^\mu(x) = 0 \iff \frac{\partial j^0}{\partial t} + \nabla \cdot \bm{j} = 0
\end{equation*}
守恒荷 $Q$ 定义为:
\begin{equation*}
    Q = \int \dif^3 x \, j^0(x), \qquad \frac{\dif Q}{\dif t} = - \int \dif^3 x \, \nabla \cdot \bm{j} = 0
\end{equation*}
\subnsection{能量-动量张量 (Energy-Momentum Tensor)}
\begin{equation*}
    T^{\mu\nu} = \frac{\partial \mathcal{L}}{\partial(\partial_\mu \phi)} \partial^\nu \phi - \mathcal{L} g^{\mu\nu}
\end{equation*}
u=0对应守恒菏,v=0对应时间,$v\neq0$对应空间
\begin{align*}
    \int \dif^3 x \, T^{00} &= \int \dif^3 x \, (\pi \dot{\phi} - \mathcal{L}) = H \quad (\text{能量}) \\
    \int \dif^3 x \, T^{0i} &= -\int \dif^3 x \, \pi \partial_i \phi = P^i \quad (\text{动量})
\end{align*}




\nsection{传播子}
\begin{equation*}
    D(x-y) = \bra{0} \phi(x)\phi(y) \ket{0} = \int \frac{\dif^3 p}{(2\pi)^3 2E_p} \eu^{-\ramuno p \cdot (x-y)} \quad (\text{洛伦兹不变量})
\end{equation*}
K-G传播子
\begin{equation*}
   D_{R} (x-y)= \theta(x^0 - y^0)\bra{0}\phi(x)\phi(y)\ket{0} =\int \frac{\dif^4 p}{(2\pi)^4} \frac{\ramuno}{p^2 - m^2} \eu^{-\ramuno p \cdot (x-y)}
\end{equation*}
\textbf{推迟格林函数 (Retarded Green's Function)}:
费曼传播子
\begin{align*}
   D_{ij}= D_F(x-y) &= \int \frac{\dif^4 p}{(2\pi)^4} \frac{\ramuno}{p^2 - m^2 + \ramuno\epsilon} \eu^{-\ramuno p \cdot (x-y)} \\
    &= \bra{0} T \phi(x)\phi(y) \ket{0} \\
\end{align*}

$D_F$和$D_R$都是格林函数:
\begin{equation*}
    (\partial^2 + m^2) D_F(x-y) = -\ramuno \delta^{(4)}(x-y)
\end{equation*}
\begin{center}
    \fbox{\parbox{0.4\textwidth}{
        \textbf{通理}:若算子 $L G = \delta$,则在任意源项 $f(x)$ 下, $L u(x) = f(x)$。解为 $u(x) = \int f(x') G(x, x') \dif x'$
    }}
\end{center}

\textbf{例:有源项的场方程,加了源$j(x)$}
其解可以写为自由场 $\phi_0$ 加上源项产生的特解:
\begin{equation*}
    \phi(x) = \phi_0(x) + \ramuno \int \dif^4 y \, D_R(x-y) j(y)\, | a_{\bm{p}} \to a_{\bm{p}} + \frac{\ramuno j(p)}{\sqrt{2E_p}}
\end{equation*}
\subnsection{Schwinger-Dyson (SD) 方程}
记号 $\braket{\dots} = \bra{\varOmega} T(\dots) \ket{\varOmega}$。对于相互作用场:
\begin{equation*}
    (\square_x + m^2) \braket{\phi_x \phi_1 \dots \phi_n} = \braket{\mathcal{L}'_{int} \phi_1 \dots \phi_n} - \ramuno \hbar \sum_j \delta^{(4)}(x-x_j) \braket{\phi_1 \dots \cancel{\phi}_j \dots \phi_n}
\end{equation*}
\subnsection{LSZ 约化公式}
\begin{align*}
    \ket{i} &= \sqrt{2\omega_1} \sqrt{2\omega_2} \, a_{\bm{p}_1}^\dagger(-\infty) a_{\bm{p}_2}^\dagger(-\infty) \ket{\varOmega} \\
    \ket{f} &= \sqrt{2\omega_3} \dots \sqrt{2\omega_n} \, a_{\bm{p}_3}^\dagger(+\infty) \dots a_{\bm{p}_n}^\dagger(+\infty) \ket{\varOmega}
\end{align*}
LSZ将S联系到关联函数
\begin{equation*}
    \braket{f | S | i} = \left[ -\ramuno \int \dif^4 x_1 \, \eu^{-\ramuno p_1 x_1} (\square_1 + m^2) \right] \dots \bra{\varOmega} T \phi(x_1) \phi(x_2) \dots \ket{\varOmega}
\end{equation*}
其中 $
    S = \bm{1} + \ramuno (2\pi)^4 \delta^{(4)}(\sum p) \mathcal{M}$
\begin{equation*}
    \bra{\varOmega} T \{ \phi \dots \} \ket{\varOmega} = \frac{\bra{0} T \{ \phi_I \dots \phi_I \exp(\ramuno \int \dif^4 x \mathcal{L}_{\text{int}}) \} \ket{0}}{\bra{0} T \{ \exp(\ramuno \int \dif^4 x \mathcal{L}_{\text{int}}) \} \ket{0}}
\end{equation*}

\subnsection{散射截面}
\begin{equation*}
    \dif \sigma = \frac{1}{(2E_1)(2E_2)|\bm{v}_1 - \bm{v}_2|} |\mathcal{M}|^2 \dif \varPi_{\text{LIPS}}
\end{equation*}
其中 $\dif \varPi_{\text{LIPS}}$ 为**洛伦兹不变相空间** (Lorentz Invariant Phase Space) 因子:
\begin{equation*}
    \dif \varPi_{\text{LIPS}} = \left( \prod_f \frac{\dif^3 p_f}{(2\pi)^3 2E_{p_f}} \right) (2\pi)^4 \delta^{(4)}(\sum p_i - \sum p_f)
\end{equation*}
质心系下的微分散射截面:
\begin{equation*}
    \left( \frac{\dif \sigma}{\dif \varOmega} \right)_{\text{CM}} = \frac{1}{64\pi^2 E_{\text{cm}}^2} \frac{|\bm{p}_f|}{|\bm{p}_i|} |\mathcal{M}|^2 \Theta(E_{\text{cm}} - m_3 - m_4)
\end{equation*}
其中总能量守恒 $E_{\text{cm}} = E_1 + E_2 = E_3 + E_4$。
\subnsection{衰变 (Decay)$1 \to n$的散射}
衰变率 $\dif \varGamma$ 定义为:
\begin{equation*}
    \dif \varGamma =\frac{\dif P}{T}= \frac{1}{2E_1} |\mathcal{M}|^2 \dif \varPi_{\text{LIPS}}
\end{equation*}



\nsection{费曼规则}
\subnsection{坐标空间}
\small{$g^2$代表有2个顶点}
\begin{itemize}
  \item  传播子:$D_F(x-y)$\,
     顶点:$\int -\ramuno g \dif^4 x$\,
    外部点:1
  \item 最后除以对称因子S
\end{itemize}
    
    
\subnsection{动量空间}
\begin{itemize}
  \item 传播子:$\frac{\ramuno}{p^2 - m^2 + \ramuno\epsilon}$\,
顶点:$\ramuno \lambda$
外部点:1
  \item 有1个圈,乘$\int \frac{\dif^4 k}{(2\pi)^4 }$(旋量再乘$-1$,标量不用管)
  \item 最后除以对称因子S
\end{itemize}

\subnsection{Scalar QED($\pi^\pm$ 介子)}
\subnsection{拉格朗日量与相互作用}
自由场拉格朗日量:
\begin{equation*}
    \mathcal{L} = -(\partial_\mu \phi)^* (\partial^\mu \phi) - m^2 \phi^* \phi - \frac{1}{4}F_{\mu\nu}^2
\end{equation*}
引入协变导数 $D_\mu = \partial_\mu + \ramuno e A_\mu$ 进行替换 $\partial_\mu \to D_\mu$:

