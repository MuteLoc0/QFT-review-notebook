\nsection{Spinors}
$\gamma$矩阵和洛仑兹代数
\begin{equation*}
    \scriptstyle\left[S^{\mu\nu},S^{\rho\sigma}\right] = \ramuno \big(
            g^{\nu\rho}S^{\mu\sigma}-g^{\mu\rho}S^{\nu\sigma}-g^{\nu\sigma}S^{\mu\rho}+g^{\mu\sigma}S^{\nu\rho}
        \big).
\end{equation*}
无限小变化${T^\mu}_\nu = {\delta^\mu}_\nu + {\theta^\mu}_\nu$下,矢量表象的洛仑兹变换满足
    \begin{equation*}
        \scriptstyle {(\varLambda_v)^\mu}_\nu = {\delta^\mu}_\nu + \ramuno \theta_{\rho\sigma}{(M^{\rho\sigma})^\mu}_\nu,
    \end{equation*}
    其中 ${(M^{\rho\sigma})^\mu}_\nu$ 也有洛仑兹代数$\scriptstyle [M^{ab},M^{cd}] = \ramuno\big(-g^{ac} M^{bd} + (a\leftrightarrow b)\big)-(c\leftrightarrow d)$,一种显示表示是$\scriptstyle{(M^{\rho\sigma})^\mu}_\nu = \ramuno(g^{\rho\mu}{\delta^\sigma}_\nu - {\delta^\rho}_\nu g^{\sigma\mu})$。

$\gamma$矩阵性质:
    \begin{align*}
        \scriptstyle \{\gamma^\mu,\gamma^\nu\} &= 2 g^{\mu\nu},\\
        \scriptstyle\gamma^\mu &\scriptstyle= \begin{pmatrix}
            0 &\sigma^\mu \\
            \bar{\sigma}^\mu &0
        \end{pmatrix},\\
        \scriptstyle(\gamma^\mu)^\dagger &\scriptstyle= \gamma^0 \gamma^\mu \gamma^0,\quad \text{where }\sigma^\mu \equiv (1, \bm{\sigma}),\qquad \bar{\sigma}^\mu \equiv (1,-\bm{\sigma}),\\
        \scriptstyle\gamma_\mu \gamma^\mu &\scriptstyle= 4,\quad\text{Tr}[\gamma^{\mu_1} \dots \gamma^{\mu_n}] = \text{Tr}[\gamma^{\mu_n} \dots \gamma^{\mu_1}]\\
        \scriptstyle\text{Tr}[\gamma^\alpha \gamma^\mu \gamma^\beta \gamma^\nu] &\scriptstyle= 4 (g^{\alpha\mu}g^{\beta\nu} + g^{\alpha\nu}g^{\mu\beta} - g^{\alpha\beta}g^{\mu\nu}),\\
        \scriptstyle\gamma_\mu \slashed{a}\slashed{b} \gamma^\mu &\scriptstyle= 4(a\cdot b)I_4,\quad \text{Tr}[\text{odd number of } \gamma] = 0\\
        \scriptstyle\gamma_\mu \slashed{a}\slashed{b}\slashed{c} \gamma^\mu &\scriptstyle= -2\slashed{c}\slashed{b}\slashed{a}.
    \end{align*}
另外$\scriptstyle (p\cdot{\sigma})(p\cdot\bar{\sigma})=m^2 \mathbb{1}_2.$

\concept{Dirac equation*}{在Weyl spinors表象下}
\begin{equation*}
        \scriptstyle(\ramuno \gamma^\mu \partial_\mu - m)\psi = \begin{pmatrix}
            -m & \ramuno \sigma\cdot \partial\\
            \ramuno \bar{\sigma}\cdot \partial & -m
        \end{pmatrix}\begin{pmatrix}
            \psi_L\\
            \psi_R
        \end{pmatrix} = 0.
\end{equation*}
据$\scriptstyle\varLambda_s^{-1}\gamma^\mu \varLambda_s = {(\varLambda_v)^\mu}_\nu \gamma^\nu$在洛仑兹变换下不变。其平面波解$\phi = u_s(p)\eu^{-\ramuno p x}$和$v_s(p)\eu^{\ramuno p x}$,$s=1,2$描述偏振方向,其中
\begin{equation*}
    \scriptstyle u_s = \begin{pmatrix}
        \scriptstyle \sqrt{p\cdot\sigma}\zeta_s \\
        \scriptstyle \sqrt{p\cdot\bar{\sigma}}\zeta_s
    \end{pmatrix} \quad v_s = \begin{pmatrix}
        \scriptstyle \sqrt{p\cdot \sigma}\eta_s\\
        \scriptstyle -\sqrt{p\cdot \bar{\sigma}}\eta_s
    \end{pmatrix}
\end{equation*}
有定义$\bar{\psi} \equiv \psi^\dagger \gamma^0$。自旋求和完整关系$\scriptstyle\sum_{s=1,2}  u_s \bar{u}_s = \gamma^\mu p_\mu + m$与$\scriptstyle\sum_{s} v^s(p) \bar{v}^s(p) = \not{p} - m$,正交关系$\scriptstyle\bar{u}_r u_s = 2m \delta_{rs}$

\begin{align*}
    \scriptstyle\psi(x) &\scriptstyle= \int \frac{d^3p}{(2\pi)^3 \sqrt{2E_p}} \sum_{s=1,2} \left( a_p^s u^s(p) e^{-ip\cdot x} + b_p^{s\dagger} v^s(p) e^{ip\cdot x} \right)\\
    \scriptstyle\bar{\psi}(x) &\scriptstyle= \int \frac{d^3p}{(2\pi)^3 \sqrt{2E_p}} \sum_{s=1,2} \left( a_p^{s\dagger} \bar{u}^s(p) e^{ip\cdot x} + b_p^s \bar{v}^s(p) e^{-ip\cdot x} \right)\\
    \scriptstyle S_F(x-y) = \int \frac{d^4p}{(2\pi)^4} \frac{i(\not{p} + m)}{p^2 - m^2 + i\epsilon} e^{-ip\cdot (x-y)}
\end{align*}

\nsection{Feynman Rule of spinor QED}
可能用到的公式\concept{Dyson series}{
    $H = H_0 + H_I$,时间演化因子$\scriptstyle U(t,t_0) = T\left\{ \exp \left[-\ramuno\int_{t_0}^t\dif t' H_I(t') \right]\right\}$并有$\scriptstyle S = \lim_{t \to \infty,\ t_0 \to -\infty} U(t, t_0) = \sum_{n=0}^{\infty} S^{(n)}$,Dyson级数展开为
    \begin{equation*}
        \scriptstyle U(t,t_0) = 1- \ramuno \int_{t_0}^t \dif t' H_I(t') - \frac{1}{2}\int_{t_0}^t \dif t' \int_{t_0}^t \dif t'' T\{H_I(t')H_I(t'')\}+ \cdots.
    \end{equation*}
}
$e^-$反子的粒子流方向和动量方向一致,而带有$e^+$电荷的正子粒子流方向和动量方向相反。

标量粒子和玻色子$\scriptstyle T\{\phi_1 \phi_2\} = \theta(x_1-x_2) \phi_1 \phi_2 + \theta(x_2-x_1) \phi_2 \phi_1$,而费米子有$\scriptstyle T\{\psi(x_1) \bar{\psi}(x_2)\} = \theta(t_1 - t_2) \psi(x_1) \bar{\psi}(x_2) - \theta(t_2 - t_1) \bar{\psi}(x_2) \psi(x_1)$
\begin{itemize}[nosep]
    \item 光子传播子:$\scalebox{0.7}{\feynmandiagram [horizontal=a to b, node distance=1cm,] {a --[photon, edge label = $p$] b};}= \textstyle \frac{-\ramuno}{p^2 +\ramuno\epsilon}\left[g_{\mu\nu} - (1-\xi)\frac{p_\mu p_\nu}{p^2}\right]$,一般在费曼规范下$\xi=1$为$\scriptstyle \frac{-\ramuno g_{\mu\nu}}{p^2+\ramuno\epsilon}$
    \item 旋量传播子:$\scalebox{0.7}{\feynmandiagram [horizontal=a to b, node distance=1cm,] {a --[fermion, edge label = $p$] b};}= \frac{\ramuno(\slashed{p}+m)}{p^2 - m^2 +\ramuno\epsilon}$
    \item 外线光子:\begin{align*}
        \scalebox{0.7}{\feynmandiagram [horizontal=a to b, node distance=1cm,] {a --[photon] b[dot]};} &\quad\scriptstyle= \epsilon_\mu (p)\quad\text{传入}\\
        \scalebox{0.7}{\feynmandiagram [horizontal=a to b, node distance=1cm,] {a[dot] --[photon] b};}&\quad\scriptstyle=\epsilon^*_\mu (p)\quad \text{传出}
    \end{align*}
    \item 外线旋量:\begin{align*}
        \scalebox{0.7}{\feynmandiagram [horizontal=a to b, node distance=1cm,] {a --[fermion] b[dot]};} &\quad\scriptstyle= u^s (p)\quad\text{传入}\\
        \scalebox{0.7}{\feynmandiagram [horizontal=a to b, node distance=1cm,] {a[dot] --[fermion] b};}&\quad\scriptstyle=\bar u^s (p)\quad \text{传出}\\
        \scalebox{0.7}{\feynmandiagram [horizontal=a to b, node distance=1cm,] {a --[anti fermion] b[dot]};} &\quad\scriptstyle= \bar v^s(p)\quad\text{正子,动量传入}\\
        \scalebox{0.7}{\feynmandiagram [horizontal=a to b, node distance=1cm,] {a[dot] --[anti fermion] b};}&\quad\scriptstyle=v^s (p)\quad \text{正子,动量传出}
    \end{align*}
    \item 顶点因子:无论粒子正负如何,总是粒子流传入传出$\scalebox{0.4}{\feynmandiagram [horizontal=a to b, node distance=1cm,baseline=a to b,every edge/.style={draw, thick}] {a --[ fermion] b,b --[fermion] c, b --[photon] d};}\scriptstyle\quad= -\ramuno e \gamma_\mu$
    \item 除以对称因子:一般不用考虑,目前能够讨论的基本都是$1$,有loop的除外。
    \item 从费曼图到$\scriptstyle\ramuno\mathcal{M}$,要反着粒子流,从出射写到顶点、回到入射,即总有$\scriptstyle\bar{u}^s\gamma^\mu u_s$类似的形式。其中粒子前后可以不同,按照粒子标记自旋。最后,计算$\scriptstyle\overline{|{\mathcal M}|^2} = \frac{1}{\text{入射粒子自旋可能数}} \sum_{\text{所有自旋}}|\mathcal{M}|^2$,如$\scriptstyle\overline{|\mathcal M|^2} = \frac{1}{4}\sum|\iam|^2 = \frac{1}{2}\sum_s \frac{1}{2}\sum_{s'}\sum_t\sum_{t'}|\iam(s,s'\to t,t')|^2$
    \item 符号:以某一费曼图作为$+$,其他费曼图检查是否有费米子交换,如$s,t$通道或$t,u$通道之间差个负号。如果存在费米子圈,则额外多一个负号。
\end{itemize}
检验用Ward等式:$\scriptstyle k_\mu \mathcal{M}^\mu = 0$把矩阵元里的光子极化向量 $\scriptstyle\epsilon_\mu(k)$ 替换为光子的动量 $\scriptstyle k_\mu$,结果必须为 0。

费曼图计算常用:
\begin{align*}
    \scriptstyle \sum_{s, s'} \left| \bar{u}(p') \gamma^0 u(p) \right|^2 & \scriptstyle= \text{Tr}\left[ (\slashed{p}' + m) \gamma^0 (\slashed{p} + m) \gamma^0 \right]= \text{Tr}\left[ \slashed{p}' \gamma^0 \slashed{p} \gamma^0 \right] + m^2 \text{Tr}[\gamma^0 \gamma^0] = \text{Tr}\left[ \gamma^\mu \gamma^0 \gamma^\nu \gamma^0 \right] p'_\mu p_\nu + 4m^2,
\end{align*}

Example: $e^- e^+ \to e^+ e^-$ Bhabha Scattering
\begin{align*}
    % s-channel diagram (Annihilation)
    \scalebox{0.3}{\feynmandiagram [vertical=p2 to p1, baseline=(current bounding box.center),every edge/.style={draw,very thick}] {
        e1 [particle = $e^-$] -- [fermion, edge label = $p$] p1, 
        e2 [particle = $e^+$] -- [anti fermion, momentum' = $k$] p1,
        p1 -- [photon, edge label = $q$] p2,
        p2 -- [fermion, edge label = $p'$] e3 [particle = $e^-$],
        p2 -- [anti fermion, momentum' = $k'$] e4 [particle = $e^+$]
    }; }
    &= \scriptstyle\ramuno\iam_s = \frac{-\ramuno g_{\mu\nu}}{(p+k)^2 + \ramuno \epsilon} 
    \left[ \bar{v}^t(k) \gamma^\mu u^s(p) \right] 
    \left[ \bar{u}^{s'}(p') \gamma^\nu v^{t'}(k') \right] (-\ramuno e)^2, \\
    % t-channel diagram (Scattering)
    \scalebox{0.3}{\feynmandiagram [horizontal=p1 to p2, baseline=(current bounding box.center),every edge/.style={draw,very thick}] {
        e1 [particle = $e^-$] -- [fermion, edge label = $p$] p1
           -- [fermion, edge label = $p'$] e3 [particle = $e^-$],
        p1 -- [photon, edge label' = $q$] p2,
        e2 [particle = $e^+$] -- [anti fermion, momentum' = $k$] p2
           -- [anti fermion, momentum' = $k'$] e4 [particle = $e^+$]
    }; }
    &= \scriptstyle\ramuno\iam_t = \frac{-\ramuno g_{\mu\nu}}{(p-p')^2 + \ramuno\epsilon} 
    \left[ \bar{u}^{s'}(p') \gamma^\mu u^s(p) \right] 
    \left[ \bar{v}^t (k) \gamma^\nu v^{t'}(k') \right] (-\ramuno e)^2.
\end{align*}
\vspace{-1.6em}
\begin{equation*}
    \scriptstyle \text{算例:}\frac{1}{4}\cancel{\sum_{\text{spins}}}\Bigg\{
        \frac{e^4}{s^2}\cancelto{\text{Tr}[(\slashed{k}-m_e)\gamma^\mu(\slashed{p}+m_e)\gamma^\nu]}{\left[\bar v^t (k) \gamma^\mu u^s(p) \bar u^s(p) \gamma^\nu v^t(k)\right]}\cancelto{\text{Tr}[(\slashed{p'}+m_e)\gamma_\mu (\slashed{k'}-m_e)\gamma_\nu]}{\left[ \bar u^{s'}(p')\gamma_\mu v^{t'}(k') \bar v^{t'}(k') \gamma_\nu u^{s'}(p')\right]} + \cdots
    \Bigg\}
\end{equation*}
本身是个标量,前面加上$\text{Tr}$求迹等价,然后改换位置套公式,每一个小区都要循环。在碰撞能量远大于电子能量时,
\begin{equation*}
    \scriptstyle s = (p+k)^2 = m_p + m_k + 2 p\cdot k \approx 2 p\cdot k = 2 p' \cdot k'
\end{equation*}

\nsection{重整化}
\concept{ $\phi^4$ 理论}{} 首先树图级,$\ramuno \mathcal{M}_1 = -\ramuno \lambda$\\
再考虑单圈级,只算$s$-channel 的单圈修正
\begin{equation*}\scriptstyle
    \ramuno \mathcal{M}_2  =\scalebox{0.5}{\begin{tikzpicture}[baseline=(current bounding box.center)]
        \begin{feynman}
            \vertex (c);
            \vertex [left=0.64cm of c] (a);
            \vertex [right=0.64cm of c] (b);
            \vertex [above left=1.36cm of c] (p1) {\(p_1\)};
            \vertex [above right=1.36cm of c] (p3) {\(p_3\)};
            \vertex [below left=1.36cm of c] (p2) {\(p_2\)};
            \vertex [below right=1.36cm of c] (p4) {\(p_4\)};
            \diagram* {
                (p1) --  (a),
                (p2) -- (a),
                (a) -- [out=71.5, in=108.5, looseness=1.5] (b),
                (a) -- [out=-71.5, in=-108.5, looseness=1.5] (b),
                (b) --  (p3),
                (b) -- (p4),
            };
        \end{feynman}
    \end{tikzpicture}} = \frac{(-\ramuno \lambda)^2}{2} \int \frac{\dif^4 k}{(2\pi)^4} \frac{\ramuno}{k^2 + \ramuno \epsilon} \frac{\ramuno}{(p-k)^2 + \ramuno \epsilon}
\end{equation*}
数学上可证明:
$
    \frac{\dif \mathcal{M}_2}{\dif s} \sim -\frac{\lambda^2}{32\pi^2} \frac{1}{s}
$
积分后得到总散射振幅:
\begin{equation*}\scriptstyle
    \mathcal{M} = \mathcal{M}_1 + \mathcal{M}_2 = -\lambda - \frac{\lambda^2}{32\pi^2} \ln \left( \frac{s}{\Lambda^2} \right)\qquad(\Lambda \text{ 是无限大的截断})
\end{equation*}
利用可观测量消除发散:
定义物理耦合常数 $\lambda_R$ 为在某一能标 $s_0$ 处测得的散射振幅:
\begin{equation*}\scriptstyle
    \lambda_R = - \mathcal{M}(s_0) = \lambda + \frac{\lambda^2}{32\pi^2} \ln \left( \frac{s_0}{\Lambda^2} \right) \quad (\text{有限值})
\end{equation*}
则$\lambda \approx \lambda_R - \frac{\lambda_R^2}{32\pi^2} \ln \left( \frac{s_0}{\Lambda^2} \right)$,
代回总振幅表达式,发散项 $\ln \Lambda^2$ 被抵消:
\begin{equation*}\scriptstyle
    \mathcal{M}(s) = 
    -\lambda_R - \frac{\lambda_R^2}{32\pi^2} \ln \left( \frac{s}{s_0} \right)
\end{equation*}
用一个可观测量($\lambda_R$)来预测另一个量.

\concept{$\psi^3$理论}{}
\begin{equation*}\scriptstyle
    \ramuno \mathcal{M}_{\text{loop}}(p) 
    =\scalebox{0.4}{\feynmandiagram [horizontal=a to b, baseline=(current bounding box.center)] {
            i1 -- [ momentum=\(p\)] a,
            b -- [momentum=\(p\)] o1,
            a -- [momentum=\(k-p\), half left] b,
            b -- [momentum=\(k\), half left] a,
            };} = \frac{1}{2}(\ramuno g)^2 \int \frac{\dif^4 k}{(2\pi)^4} \frac{\ramuno}{(k-p)^2 - m^2 + \ramuno\epsilon} \frac{\ramuno}{k^2 - m^2 + \ramuno\epsilon}
\end{equation*}
费曼参数化(Feynman Parameters)变分母 $\frac{1}{AB} = \int_0^1 \dif x \frac{1}{[A+(B-A)x]^2}$,然后做平移:
$k^\mu \to k^\mu + p^\mu(1-x)$,定义:$\Delta = m^2 - p^2 x(1-x)$。
积分变为:
\begin{equation*}\scriptstyle
    \ramuno \mathcal{M}_{\text{loop}}(p) = \frac{g^2}{2} \int \frac{\dif^4 k}{(2\pi)^4} \int_0^1 \dif x \frac{1}{[k^2 - \Delta + \ramuno\epsilon]^2}
\end{equation*}
接着使用Pauli–Villars formula$\scriptstyle \int \frac{\dif^4 k}{(2\pi)^4} \frac{1}{(k^2 - \Delta + \ramuno\varepsilon)^2} = - \frac{\ramuno}{16\pi^2} \ln \frac{\Delta}{\Lambda^2}.$,并假设m=0,得到:
\begin{equation*}\scriptstyle
    \mathcal{M}_{\text{loop}}(p) = -\frac{g^2}{32\pi^2} \ln \frac{Q^2}{\Lambda^2}\qquad( Q^2 = -p^2 > 0)
\end{equation*}
加上没有圈图的情况,有:
\begin{equation*}\scriptstyle
  \mathcal{M}(Q) = \mathcal{M}^0(Q) + \mathcal{M}^1(Q) = \frac{g^2}{Q^2} \left( 1 - \frac{1}{32\pi^2} \frac{g^2}{Q^2} \ln \frac{Q^2}{\Lambda^2} + \mathcal{O}(g^4) \right)
\end{equation*}
用观测量来表示$\tilde{g}_R^2 = \mathcal{M}(Q_0) = \tilde{g}^2 - \frac{1}{32\pi^2} \tilde{g}^4 \ln \frac{Q_0^2}{\Lambda^2} + \mathcal{O}(\tilde{g}^6)$,最终得到:
\begin{equation*}\scriptstyle
  \mathcal{M}(Q)  = \tilde{g}_R^2 + \frac{1}{32\pi^2} \tilde{g}_R^4 \ln \frac{Q_0^2}{Q^2} + \mathcal{O}(\tilde{g}_R^6)
\end{equation*}
\concept{维数正规化}{
    \begin{align*}
        \scriptstyle\int \frac{\dif^d l}{(2\pi)^d} \frac{l^2}{(l^2 - \Delta + \ramuno \epsilon)^2} &=\scriptstyle \frac{d}{2} \frac{\ramuno}{(4\pi)^{d/2}}\Delta^{\frac{d}{2} - 1} \Gamma\left(\frac{2-d}{2}\right),\\
        \scriptstyle\int\frac{\dif^d l}{(2\pi)^d} \frac{1}{(l^2 - \Delta + \ramuno \epsilon)^2} &=\scriptstyle \frac{\ramuno}{(4\pi)^{d/2}}\Delta^{\frac{d}{2}-2}\Gamma\left(\frac{4-d}{2}\right).
    \end{align*}}

TODO:1. 符号 2. 旋量传播子
3. 重整化第三部分 4. 光子场