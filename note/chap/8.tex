\nsection{Spinors}
$\gamma$矩阵和洛仑兹代数
\begin{equation}
    \scriptstyle\left[S^{\mu\nu},S^{\rho\sigma}\right] = \ramuno \big(
            g^{\nu\rho}S^{\mu\sigma}-g^{\mu\rho}S^{\nu\sigma}-g^{\nu\sigma}S^{\mu\rho}+g^{\mu\sigma}S^{\nu\rho}
        \big).
\end{equation}
无限小变化${T^\mu}_\nu = {\delta^\mu}_\nu + {\theta^\mu}_\nu$下,矢量表象的洛仑兹变换满足
    \begin{equation}
        \scriptstyle {(\varLambda_v)^\mu}_\nu = {\delta^\mu}_\nu + \ramuno \theta_{\rho\sigma}{(M^{\rho\sigma})^\mu}_\nu,
    \end{equation}
    其中 ${(M^{\rho\sigma})^\mu}_\nu$ 也有洛仑兹代数$\scriptstyle [M^{ab},M^{cd}] = \ramuno\big(-g^{ac} M^{bd} + (a\leftrightarrow b)\big)-(c\leftrightarrow d)$,一种显示表示是$\scriptstyle{(M^{\rho\sigma})^\mu}_\nu = \ramuno(g^{\rho\mu}{\delta^\sigma}_\nu - {\delta^\rho}_\nu g^{\sigma\mu})$。

$\gamma$矩阵性质:
    \begin{align*}
        \{\gamma^\mu,\gamma^\nu\} &= 2 g^{\mu\nu},\\
        \scriptstyle\gamma^\mu &\scriptstyle= \begin{pmatrix}
            0 &\sigma^\mu \\
            \bar{\sigma}^\mu &0
        \end{pmatrix}, \text{where }\sigma^\mu \equiv (1, \bm{\sigma}),\qquad \bar{\sigma}^\mu \equiv (1,-\bm{\sigma}),\\
        \scriptstyle(\gamma^\mu)^\dagger &\scriptstyle= \gamma^0 \gamma^\mu \gamma^0,\\
        \scriptstyle\gamma_\mu \gamma^\mu &\scriptstyle= 4,\\
        \scriptstyle\text{Tr}[\gamma^\alpha \gamma^\mu \gamma^\beta \gamma^\nu] &\scriptstyle= 4 (g^{\alpha\mu}g^{\beta\nu} + g^{\alpha\nu}g^{\mu\beta} - g^{\alpha\beta}g^{\mu\nu}),\\
        \scriptstyle\gamma_\mu \slashed{a}\slashed{b} \gamma^\mu &\scriptstyle= 4(a\cdot b)I_4,\\
        \scriptstyle\gamma_\mu \slashed{a}\slashed{b}\slashed{c} \gamma^\mu &\scriptstyle= -2\slashed{c}\slashed{b}\slashed{a}.
    \end{align*}
另外$\scriptstyle (p\cdot{\sigma})(p\cdot\bar{\sigma})=m^2 \mathbb{1}_2.$

\concept{Dirac equation}{在Weyl spinors表象下}
\begin{equation*}
        \scriptstyle(\ramuno \gamma^\mu \partial_\mu - m)\psi = \begin{pmatrix}
            -m & \ramuno \sigma\cdot \partial\\
            \ramuno \bar{\sigma}\cdot \partial & -m
        \end{pmatrix}\begin{pmatrix}
            \psi_L\\
            \psi_R
        \end{pmatrix} = 0.
\end{equation*}
据$\scriptstyle\varLambda_s^{-1}\gamma^\mu \varLambda_s = {(\varLambda_v)^\mu}_\nu \gamma^\nu$在洛仑兹变换下不变。其平面波解$\phi = u_s(p)\eu^{-\ramuno p x}$和$v_s(p)\eu^{\ramuno p x}$,$s=1,2$描述偏振方向,其中
\begin{equation}
    \scriptstyle u_s = \begin{pmatrix}
        \scriptstyle \sqrt{p\cdot\sigma}\zeta_s \\
        \scriptstyle \sqrt{p\cdot\bar{\sigma}}\zeta_s
    \end{pmatrix} \quad v_s = \begin{pmatrix}
        \scriptstyle \sqrt{p\cdot \sigma}\eta_s\\
        \scriptstyle -\sqrt{p\cdot \bar{\sigma}}\eta_s
    \end{pmatrix}
\end{equation}
同时$\bar{\psi} \equiv \psi^\dagger \gamma^0$

\nsection{Feynman Rule of spinor QED}
可能用到的公式\concept{Dyson series}{
    $H = H_0 + H_I$,时间演化因子$\scriptstyle U(t,t_0) = T\left\{ \exp \left[-\ramuno\int_{t_0}^t\dif t' H_I(t') \right]\right\}$并有$\scriptstyle S = \lim_{t \to \infty,\ t_0 \to -\infty} U(t, t_0) = \sum_{n=0}^{\infty} S^{(n)}$,Dyson级数展开为
    \begin{equation*}
        \scriptstyle U(t,t_0) = 1- \ramuno \int_{t_0}^t \dif t' H_I(t') - \frac{1}{2}\int_{t_0}^t \dif t' \int_{t_0}^t \dif t'' T\{H_I(t')H_I(t'')\}+ \cdots.
    \end{equation*}
}
$e^-$反子的粒子流方向和动量方向一致,而带有$e^+$电荷的正子粒子流方向和动量方向相反。
\begin{itemize}[nosep]
    \item 光子传播子:$\scalebox{0.7}{\feynmandiagram [horizontal=a to b, node distance=1cm,] {a --[photon, edge label = $p$] b};}= \textstyle \frac{-\ramuno}{p^2 +\ramuno\epsilon}\left[g_{\mu\nu} - (1-\xi)\frac{p_\mu p_\nu}{p^2}\right]$,一般在费曼规范下$\xi=1$为$\scriptstyle \frac{-\ramuno g_{\mu\nu}}{p^2+\ramuno\epsilon}$
    \item 旋量传播子:$\scalebox{0.7}{\feynmandiagram [horizontal=a to b, node distance=1cm,] {a --[fermion, edge label = $p$] b};}= \frac{\ramuno(\slashed{p}+m)}{p^2 - m^2 +\ramuno\epsilon}$
    \item 外线光子:\begin{align}
        \scalebox{0.7}{\feynmandiagram [horizontal=a to b, node distance=1cm,] {a --[photon] b[dot]};} &\scriptstyle= \epsilon_\mu (p)\quad\text{传入}\\
        \scalebox{0.7}{\feynmandiagram [horizontal=a to b, node distance=1cm,] {a[dot] --[photon] b};}&\scriptstyle=\epsilon^*_\mu (p)\quad \text{传出}
    \end{align}
    \item 外线旋量:\begin{align}
        \scalebox{0.7}{\feynmandiagram [horizontal=a to b, node distance=1cm,] {a --[fermion] b[dot]};} &\scriptstyle= u^s (p)\quad\text{传入}\\
        \scalebox{0.7}{\feynmandiagram [horizontal=a to b, node distance=1cm,] {a[dot] --[fermion] b};}&\scriptstyle=\bar u^s (p)\quad \text{传出}\\
        \scalebox{0.7}{\feynmandiagram [horizontal=a to b, node distance=1cm,] {a --[anti fermion] b[dot]};} &\scriptstyle= \bar v^s(p)\quad\text{动量传入}\\
        \scalebox{0.7}{\feynmandiagram [horizontal=a to b, node distance=1cm,] {a[dot] --[anti fermion] b};}&\scriptstyle=v^s (p)\quad \text{动量传出}
    \end{align}
\end{itemize}
Example: $e^- e^+ \to e^+ e^-$ Bhabha Scattering
\begin{align*}
    % s-channel diagram (Annihilation)
    \scalebox{0.5}{\feynmandiagram [vertical=p2 to p1, baseline=(current bounding box.center),every edge/.style={draw, thick}] {
        e1 [particle = $e^-$] -- [fermion, edge label = $p$] p1, 
        e2 [particle = $e^+$] -- [anti fermion, momentum' = $k$] p1,
        p1 -- [photon, edge label = $q$] p2,
        p2 -- [fermion, edge label = $p'$] e3 [particle = $e^-$],
        p2 -- [anti fermion, momentum' = $k'$] e4 [particle = $e^+$]
    }; }
    &= \scriptstyle\ramuno\iam_s = \frac{-\ramuno g_{\mu\nu}}{(p+k)^2 + \ramuno \epsilon} 
    \left[ \bar{v}^t(k) \gamma^\mu u^s(p) \right] 
    \left[ \bar{u}^{s'}(p') \gamma^\nu v^{t'}(k') \right] (-\ramuno e)^2, \\
    % t-channel diagram (Scattering)
    \scalebox{0.5}{\feynmandiagram [horizontal=p1 to p2, baseline=(current bounding box.center),every edge/.style={draw, thick}] {
        e1 [particle = $e^-$] -- [fermion, edge label = $p$] p1
           -- [fermion, edge label = $p'$] e3 [particle = $e^-$],
        p1 -- [photon, edge label' = $q$] p2,
        e2 [particle = $e^+$] -- [anti fermion, momentum' = $k$] p2
           -- [anti fermion, momentum' = $k'$] e4 [particle = $e^+$]
    }; }
    &= \scriptstyle\ramuno\iam_t = \frac{-\ramuno g_{\mu\nu}}{(p-p')^2 + \ramuno\epsilon} 
    \left[ \bar{u}^{s'}(p') \gamma^\mu u^s(p) \right] 
    \left[ \bar{v}^t (k) \gamma^\nu v^{t'}(k') \right] (-\ramuno e)^2.
\end{align*}

TODO:1. 符号 2. 旋量传播子
3. 重整化第三部分 4. 光子场