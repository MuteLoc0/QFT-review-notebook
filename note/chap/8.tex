\nsection{Spinors}
\nsection{Feynman Rule of spinor QED}
Example: $e^- e^+ \to e^+ e^-$ Bhabha Scattering
\begin{align}
    % s-channel diagram (Annihilation)
    \scalebox{0.5}{\feynmandiagram [vertical=p2 to p1, baseline=(current bounding box.center),every edge/.style={draw, thick}] {
        e1 [particle = $e^-$] -- [fermion, edge label = $p$] p1, 
        e2 [particle = $e^+$] -- [anti fermion, momentum' = $k$] p1,
        p1 -- [photon, edge label = $q$] p2,
        p2 -- [fermion, edge label = $p'$] e3 [particle = $e^-$],
        p2 -- [anti fermion, momentum' = $k'$] e4 [particle = $e^+$]
    }; }
    &= \scriptstyle\ramuno\iam_s = \frac{-\ramuno g_{\mu\nu}}{(p+k)^2 + \ramuno \epsilon} 
    \left[ \bar{v}^t(k) \gamma^\mu u^s(p) \right] 
    \left[ \bar{u}^{s'}(p') \gamma^\nu v^{t'}(k') \right] (-\ramuno e)^2, \\
    % t-channel diagram (Scattering)
    \scalebox{0.5}{\feynmandiagram [horizontal=p1 to p2, baseline=(current bounding box.center),every edge/.style={draw, thick}] {
        e1 [particle = $e^-$] -- [fermion, edge label = $p$] p1
           -- [fermion, edge label = $p'$] e3 [particle = $e^-$],
        p1 -- [photon, edge label' = $q$] p2,
        e2 [particle = $e^+$] -- [anti fermion, momentum' = $k$] p2
           -- [anti fermion, momentum' = $k'$] e4 [particle = $e^+$]
    }; }
    &= \scriptstyle\ramuno\iam_t = \frac{-\ramuno g_{\mu\nu}}{(p-p')^2 + \ramuno\epsilon} 
    \left[ \bar{u}^{s'}(p') \gamma^\mu u^s(p) \right] 
    \left[ \bar{v}^t (k) \gamma^\nu v^{t'}(k') \right] (-\ramuno e)^2.
\end{align}