
\nsection{Feynman Rule of Scalar Field}
费曼规则解决$S_{fi} = \braket{f|S|i} \sim \braket{\Omega|T\{\phi(x_1)\cdots \phi(x_n)\}|}$后面式子的计算,$\ket{\Omega}$是相互作用理论的基态/真空,$\ket{0}$是自由理论的基态。
\begin{equation}
    D_F(x,y) \equiv \braket{0|T\{\phi_0(x)\phi_0(y)\}} = \int \frac{\dif ^4 k}{(2\pi)^4} \frac{\ramuno}{k^2- m^2 + \ramuno\epsilon}\eu^{\ramuno k (x-y)}
\end{equation}
Schwartz书中\concept{Schwinger-Dyson equations}{
\begin{equation}
    (\square_x + m^2)\braket{\phi_x \phi_1 \cdots \phi_n} = \braket{\mathcal{L}'_\text{int}[\phi_x]\phi_1 \cdots \phi_n} - \ramuno\hbar \sum_j \delta^4(x-x_j)\braket{\text{没有}\phi_j}
\end{equation}}
这里$\braket{\cdots} = \braket{\Omega| T\{\cdots\}|\Omega}$,以及相应的$m=0$(为了描述简便。$m\neq0$是简单的推广)下的$\square_x D_{x1} = -\ramuno \delta_{x1}$,$\braket{\phi_1 \phi_2} = \int \dif^4 x \, D_{x_1} \square_x\braket{\phi_x \phi_2}$。

Wick定理,时序算符和收缩的来源:
\begin{equation}
    \braket{\Omega|T\{\phi(x_1)\cdots \phi(x_n)\}|\Omega} = \frac{\braket{0|T\{\phi_0(x_1)\cdots \phi_0(x_n)\eu^{\ramuno\int \dif^4x\,\mathcal{L}_\text{int}[\phi_0]}\}|0}}{\braket{0|T\{\eu^{\ramuno\int \dif^4x\,\mathcal{L}_\text{int}[\phi_0]}\}|0}}
\end{equation}
\concept{Wick's Theorem}{
    真空场正负展开$\phi_0(x) = \phi_+(x) + \phi_-(x)$其中$\phi_\pm = \int \frac{\dif^3 p}{(2\pi)^3}\frac{1}{2\sqrt{2\omega_p}}a^{\pm}_p\eu^{\pm\ramuno px}$最后$\braket{0|T\{\cdots\}|0}$值只会在所有都\textbf{收缩}下存在。\concept{收缩}{$\phi_+$和$\phi_-$配对。}
}

\concept{对称因子的计算}{1. 如果一个传播子首尾连接于一个顶点,那么贡献对称因子2;2. 如果两个点之间连接$n$个全同传播子,那么贡献对称因子$n!$;3. 如果费曼图中有一个$n$阶旋转对称轴,即存在一个$n$阶置换对称不变性,贡献$n!$}

\concept{相互作用顶点}{接上的线的数量取决于相互作用项,如$\phi^4$理论就能接入4条线,又如$\frac{-g}{1\times2!}\Phi\phi^2$理论接线数是1条$\Phi$线和2条$\phi$线。}

\concept{位置空间的费曼规则}{}以$\mathcal{L}_\text{int} = -\frac{\lambda}{4!}\phi^4$理论为例------\vspace{-1em}
\begin{itemize}[nosep]
    \item 内部传播子:$\scalebox{0.7}{\feynmandiagram [horizontal=a to b, node distance=1cm] {
        a [dot, label=left:$x$] -- b [dot, label=right:$y$]
    };} = D_F(x-y)$
    \item 顶点:$\scalebox{0.7}{\feynmandiagram [node distance=1cm] {
        a [dot, label=left:$z$]
    };} = (-\ramuno\lambda)\int \dif^4 z$
    \item 外线:$\scalebox{0.7}{\feynmandiagram [horizontal=a to b,node distance=1cm] {
        a [dot, label=left:$z$]--b
    };} = 1$
    \item 除以对称因子
\end{itemize}
\concept{动量空间$\phi^4$费曼规则}{\vspace{-1em}
    \begin{itemize}[nosep]
        \item 传播子:$\scalebox{0.7}{\feynmandiagram [horizontal=a to b, node distance=1cm,] {a --[fermion, edge label = $p$] b};}= \frac{\ramuno}{p^2 - m^2 +\ramuno\epsilon}$
        \item 对于每个相互作用顶点:$\scalebox{0.35}{\feynmandiagram [horizontal=a to b, node distance=1cm,baseline=(current bounding box.center)] {a --[fermion]f[dot],b --[fermion]f,c --[fermion]f,d --[fermion]f};}= -\ramuno \lambda$
        \item 外点:$\scalebox{0.7}{\feynmandiagram [horizontal=a to b, node distance=1cm, baseline = a to b] {b --[fermion, edge label = $p$] a[label = above:$x$,dot]};}= \eu^{-\ramuno p\cdot x}$
        \item 对每个相互作用顶点应用动量守恒$\delta(p + k \cdots)$
        \item 对所有内动量积分$\int \frac{\dif^4 p}{(2\pi)^4}$
        \item 除以对称因子
    \end{itemize}
}
对一个任意给出的标量场相互作用如$\mathcal{L}_\text{int} = -g \phi_1^n \phi_2^m$,如何写出其动量空间费曼规则?确定该项相互作用顶点因子:$\mathcal{L}_\text{int}=-\frac{n! m! g}{n! m!}\phi_1^n \phi_2^m$,可见顶点因子$-\ramuno n! m! g$,每个这类顶点必须接入$n$个$\phi_1$,$m$个$
\phi_2$;其余传播子形式等和上面一致。

如果存在多个相互作用项,如何画费曼图?对每个作用项画出相应的顶点;确定要画什么费曼图,如$\braket{\phi\phi\phi\phi}$是四点函数,费曼图中有四个$\phi$外线。其0阶(即自由场无相互作用)为$D_{12}D_{34} + D_{13}D_{24} + D_{14}D_{23}$;1阶是树级的,在作用项$-\frac{g}{3!}\phi^3$理论中就是$s,t,u$通道对应的图象。这里给出$s,t,u$通道费曼图,对于$\mathcal{L} = - \frac{1}{2}\phi \square \phi - \frac{1}{2}m^2 \phi^2 + \frac{g}{3!} \phi^3$,考虑$\phi\phi\to \phi\phi$散射,

\begin{center}
\begin{tabular}{c|c|c}
    \scalebox{0.5}{\begin{tikzpicture}[baseline=(current bounding box.center),every edge/.style={draw, thick}]\begin{feynman}
            \vertex (c);
            \vertex [left=0.64cm of c] (a);
            \vertex [right=0.64cm of c] (b);
            \vertex [above left=1.36cm of c] (p1) {$1$};
            \vertex [above right=1.36cm of c] (p3) {$3$};
            \vertex [below left=1.36cm of c] (p2) {$2$};
            \vertex [below right=1.36cm of c] (p4) {$4$};
            \diagram* {
                (p1) --  (a),
                (p2) -- (a),
                (a) -- (b),
                (b) --  (p3),
                (b) -- (p4),
    };\end{feynman}\end{tikzpicture}} & 
    \scalebox{0.5}{\begin{tikzpicture}[baseline=(current bounding box.center),every edge/.style={draw, thick}]\begin{feynman}
            \vertex (c);
            \vertex [above=0.64cm of c] (a);
            \vertex [below=0.64cm of c] (b);
            \vertex [above left=1.36cm of c] (p1) {$1$};
            \vertex [above right=1.36cm of c] (p3) {$3$};
            \vertex [below left=1.36cm of c] (p2) {$2$};
            \vertex [below right=1.36cm of c] (p4) {$4$};
            \diagram* {
                (p1) --  (a),
                (p3) -- (a),
                (a) -- (b),
                (b) --  (p2),
                (b) -- (p4),
    };\end{feynman}\end{tikzpicture}} & 
    \scalebox{0.5}{\begin{tikzpicture}[baseline=(current bounding box.center),every edge/.style={draw, thick}]\begin{feynman}
            \vertex (c);
            \vertex [above=0.64cm of c] (a);
            \vertex [below=0.64cm of c] (b);
            \vertex [above left=1.36cm of c] (p1) {$1$};
            \vertex [above right=1.36cm of c] (p3) {$3$};
            \vertex [below left=1.36cm of c] (p2) {$2$};
            \vertex [below right=1.36cm of c] (p4) {$4$};
            \diagram* {
                (p1) --  (a),
                (p4) -- (a),
                (a) -- (b),
                (b) --  (p2),
                (b) -- (p3),
    };\end{feynman}\end{tikzpicture}} \\
    \scalebox{0.8}{$s=(p_1+p_2)^2 = (p_3+p_4)^2$} &\scalebox{0.8}{$t=(p_1-p_3)^2 = (p_2-p_4)^2$}&\scalebox{0.8}{$\quad u=(p_1-p_4)^2 = (p_2-p_3)^2$}
\end{tabular}
\end{center}

\nsection{Feynman Rule of Scalar QED}
\begin{equation}
    F_{\mu\nu} = \begin{pmatrix}
        0& E_x& E_y& E_z\\
        -E_x &0 & -B_z & B_y\\
        -E_y & B_z &0 &-B_x \\
        -E_z & -B_y & B_x &0
    \end{pmatrix} \quad \bm{E} = - \nabla \phi - \frac{\partial \bm{A}}{\partial t}\quad \bm{B} = \nabla\times \bm A
\end{equation}
\begin{align}
    \mathcal{L} &= -\frac{1}{4}F_{\mu\nu}^2 + |D_mu \phi|^2 - m^2 |\phi|^2\quad (\text{where }D_\mu = \partial_\mu + \ramuno A_\mu)\\
    &= -\frac{1}{4}F_{\mu\nu}^2 - \phi^*(\square + m^2)\phi - \ramuno e A_\mu [\phi^*(\partial_\mu\phi) - (\partial_\mu\phi^*)\phi] + e^2 A_\mu^2|\phi|^2.
\end{align}

费曼规则:\vspace{-1em}
\begin{itemize}[nosep]
    \item 复标量传播子:$\scalebox{0.7}{\feynmandiagram [horizontal=a to b, node distance=1cm,every edge/.style={draw, thick}] {a --[dashed, fermion, edge label = $p$] b};}= \frac{\ramuno}{p^2 - m^2 +\ramuno\epsilon}$
    \item 光子传播子:$\scalebox{0.7}{\feynmandiagram [horizontal=a to b, node distance=1cm,] {a --[photon, edge label = $p$] b};}= \frac{\ramuno}{p^2 - m^2 +\ramuno\epsilon}$
    \item $-\ramuno e A_\mu[\cdots]$项顶点:标量QED顶点给出的是粒子流箭头(即线上箭头)指向外侧的粒子的动量之和减去箭头指向内侧的粒子的动量的$-\ramuno e$倍。$\scalebox{0.4}{\feynmandiagram [horizontal=a to b, node distance=1cm,baseline=a to b,every edge/.style={draw, thick}] {a[particle = $e^-$] --[dashed, fermion, momentum' =$p_1$] b,b --[dashed, fermion,momentum' = $p_2$] c[particle = $e^-$], b --[photon] d};}= \frac{\ramuno}{p^2 - m^2 +\ramuno\epsilon}$
    \item $e^2 A_\mu^2 |\phi|^2$项顶点:$\scalebox{0.4}{\begin{tikzpicture}[baseline=(current bounding box.center),every edge/.style={draw, thick}]
  \begin{feynman}
    \vertex (b);
    \vertex [above left=1.2cm of b] (a) {$e^-$};
    \vertex [above right=1.2cm of b] (c) {$e^-$};
    \vertex [below left=1.2cm of b] (d);
    \vertex [below right=1.2cm of b] (e);
    \diagram* {
      (a) -- [dashed, fermion] (b),
      (b) -- [dashed, fermion] (c),
      (b) -- [photon] (d),
      (b) -- [photon] (e),
    };
  \end{feynman}
\end{tikzpicture}}= 2\ramuno e^2 g_{\mu\nu}$
\end{itemize}

\begin{align}
    % s-channel diagram (Annihilation)
    \feynmandiagram [vertical=p2 to p1, baseline=(current bounding box.center)] {
        e1 [particle = $e^-$] -- [fermion, edge label = $p$] p1, 
        e2 [particle = $e^+$] -- [anti fermion, momentum' = $k$] p1,
        p1 -- [photon, edge label = $q$] p2,
        p2 -- [fermion, edge label = $p'$] e3 [particle = $e^-$],
        p2 -- [anti fermion, momentum' = $k'$] e4 [particle = $e^+$]
    }; 
    &= \ramuno\iam_s = \frac{-\ramuno g_{\mu\nu}}{(p+k)^2 + \ramuno \epsilon} 
    \left[ \bar{v}^t(k) \gamma^\mu u^s(p) \right] 
    \left[ \bar{u}^{s'}(p') \gamma^\nu v^{t'}(k') \right] (-\ramuno e)^2, \\[2em]
    % t-channel diagram (Scattering)
    \feynmandiagram [horizontal=p1 to p2, baseline=(current bounding box.center)] {
        e1 [particle = $e^-$] -- [fermion, edge label = $p$] p1
           -- [fermion, edge label = $p'$] e3 [particle = $e^-$],
        p1 -- [photon, edge label' = $q$] p2,
        e2 [particle = $e^+$] -- [anti fermion, momentum' = $k$] p2
           -- [anti fermion, momentum' = $k'$] e4 [particle = $e^+$]
    }; 
    &= \ramuno\iam_t = \frac{-\ramuno g_{\mu\nu}}{(p-p')^2 + \ramuno\epsilon} 
    \left[ \bar{u}^{s'}(p') \gamma^\mu u^s(p) \right] 
    \left[ \bar{v}^t (k) \gamma^\nu v^{t'}(k') \right] (-\ramuno e)^2.
\end{align}

